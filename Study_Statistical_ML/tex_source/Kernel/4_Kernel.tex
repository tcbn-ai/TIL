\documentclass[dvipdfmx]{jsarticle}
\usepackage{graphicx, url, algorithm, algorithmic, float, booktabs, listings, color, pdfpages, amsmath, amssymb, latexsym, mathtools, ascmac, amsthm}
\lstset{
  basicstyle={\ttfamily},
  identifierstyle={\small},
  commentstyle={\smallitshape},
  keywordstyle={\small\bfseries},
  ndkeywordstyle={\small},
  stringstyle={\small\ttfamily},
  frame={tb},
  breaklines=true,
  columns=[l]{fullflexible},
  numbers=left,
  xrightmargin=0zw,
  xleftmargin=3zw,
  numberstyle={\scriptsize},
  stepnumber=1,
  numbersep=1zw,
  lineskip=-0.5ex
}
\usepackage[table,xcdraw]{xcolor}
\newtheorem{theo}{定理}
\newtheorem{defi}{定義}
\newtheorem{lemm}{補題}
\newcommand{\red}[1]{\textcolor{red}{#1}}
\newcommand{\blue}[1]{\textcolor{blue}{#1}}
\newcommand{\green}[1]{\textcolor{green}{#1}}
\renewcommand{\baselinestretch}{1.1}
\usepackage{bm}
\usepackage[hang,small,bf]{caption}
\usepackage[subrefformat=parens]{subcaption}
\def\qed{\hfill $\Box$}

\newcommand{\argmin}{\mathop{\rm argmin}\limits}
% 数式番号に章番号を追加
%\makeatletter
%\@addtoreset{equation}{section}
%\def\theequation{\thesection.\arabic{equation}}
%\makeatother

\title{統計的学習理論 第4章 補足資料}
\author{}
\date{}
\begin{document}
\maketitle
スライドに載せていない証明の部分の補足資料となっている.

\paragraph{定理4.2の証明}
定理4.2を示すために,次の補題を示す.
\begin{lemm}\label{lemm:1}
  Hermite行列$A, B$が非負定値であれば,Hadamard積$A \circ B$も非負定値である.
\end{lemm}

\begin{proof}
  $A, B \in \mathbb{C}^{n \times n}$とする.
  $A, B$はHermiteより,固有値をそれぞれ$\lambda_i, \mu_i \ (i = 1, \ldots, n)$,固有ベクトルをそれぞれ$u_i, v_i \ (i = 1, \ldots, n)$とすると,$\lambda_i, \mu_i \in \mathbb{R}$であり,$u_i$ (および$v_i$) は互いに直交するようにとれる.このとき,
  \begin{align}
    A = \sum_{i = 1}^n \lambda_i u_i u_i^{*}, \ B = \sum_{j = 1}^n \mu_j v_j v_j^{*} \nonumber
  \end{align}
  とスペクトル分解できる ($u_i^{*}, v_j^{*}$は$u_i, v_j$の共役転置).ここで,
  \begin{align}
    (u_i u_i^{*}) \circ (v_j v_j^{*}) = (u_{i,k} \bar{u}_{i, l} \cdot v_{j,k} \bar{v}_{j,l})_{k,l} = (u_{i,k}v_{j,k} \cdot \bar{u}_{i,l}\bar{v}_{j,l})_{k, l}
    = (u_i \circ v_j) (u_i \circ v_j)^{*} \nonumber
  \end{align}
  が成立する.
  よって,
  \begin{align}
    A \circ B = \sum_{i = 1}^n \sum_{j = 1}^n \lambda_i \mu_j (u_i u_i^{*}) \circ (v_j v_j^{*})
    = \sum_{i = 1}^n \sum_{j = 1}^n \lambda_i \mu_j (u_i \circ v_j) (u_i \circ v_j)^{*} \nonumber
  \end{align}
  が成立する.
  $A, B$は非負定値より,$\lambda_i \geq 0, \mu_i \geq 0 \ (i = 1, \ldots, n)$であり,行列$(u_i \circ v_j) (u_i \circ v_j)^{*}$は非負定値である.
  よって,$\forall x \in \mathbb{C}^{n \times n}$に対して,
  \begin{align}
    x^{*} (A \circ B) x = \sum_{i = 1}^n \sum_{j = 1}^n \lambda_i \mu_j x^{*} (u_i \circ v_j) (u_i \circ v_j)^{*}  x \geq 0 \nonumber
  \end{align}
  が成り立ち,$A \circ B$が非負定値であることが示される.
\end{proof}

この結果を用いて,定理4.2を証明する.
\begin{proof}
  \subparagraph{1の証明}
  カーネル関数の定義において$n = 1$とすると,定義より明らか.

  \subparagraph{2の証明}
  $a k_1 + b$, $k_1 + k_2$の対称非負定値性は容易に分かる.
  カーネル関数の積$k = k_1 \cdot k_2$について考える.$x_1, \ldots, x_n \in \mathcal{X}$に対してカーネル関数$k_i$から定義されるグラム行列を$K_i$とする.このとき,$K = (k(x_i, x_j))$は,
  $K = K_1 \circ K_2$と,非負定値行列のHadamard積を用いて表される.定理\ref{lemm:1}より,$K$が非負定値であることが示され,$k_1 \cdot k_2$がカーネル関数であることが示される.

  \subparagraph{3の証明}
  極限$k_{\infty}$の対称性は明らか.ここで,各$\ell$に対して$k_{\ell}$がカーネル関数より,各$\ell$に対して
  $\sum_{i = 1}^n \sum_{j = 1}^n c_i c_j k_{\ell}(x_i, x_j) \geq 0$が成り立つ.
  よって,$\ell \to \infty$として
  $\sum_{i = 1}^n \sum_{j = 1}^n c_i c_j k_{\infty}(x_i, x_j) \geq 0$
  が示され,$k_{\infty}$の非負定値性が示される.
  よって,$k_{\infty}$がカーネル関数であることが示される.
\end{proof}



\end{document}
